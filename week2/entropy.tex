\documentclass[UTF8]{ctexart}
   \usepackage{amsmath}
   \usepackage{amssymb}
\begin{document}
    \section{} 
    \subsection{信息量}
    信息量作为信息的度量,可以用来衡量熵的定义,设 $p(x_i)$ 表示$x_i$ 发生的概率,则信息量可以表示为:
     \begin{equation}
      h(x_i) = −log_{a}p(x_i) = log_{a}\frac{1}{p(x_i)}
     \end{equation}
    $a$值常取2,表示比特,即非0即1,由此可知,信息量与概率成反比.

    \subsection{熵} 
    熵在热力学熵用来描述物质的混乱程度,用来衡量不确定性,物质越混乱,不确定性越大,熵值越大。
    到信息论中,事件发生的不确定行越大,则熵越大。
    例如:掷骰子,六个面机会均等,因此投一次得到的点数不确定性最大(因为每个点数的概率都是六分之一),因此此时熵最大;
    熵是信息量的期望,公式如下:
    \begin{equation}
    H(X) = -\sum^n_{i=1}p(x_i)log_{a}p(x_i)
    \end{equation}
    表示信息量的期望,反应不确定性。

    \subsection{联合熵}
    联合熵可以表示为两个事件的熵的并集:
    \begin{equation}
    H(X, Y) = -\sum^n_{i=1}\sum^n_{j=1}p(x_i, y_i)log_{a}p(x_i, y_j)
    \end{equation}
    其中  $max(H(X), H(Y)) <= H(X, Y) <= H(X) + H(Y)$

    \subsection{互信息}
    互信息是用来表示变量间相互以来的程度,常用在特征选择和特征关联性等方面,公式如下:
    \begin{equation}
    I(X, Y) = -\sum^n_{i=1}\sum^n_{j=1}p(x_i, y_i)log_{a}\frac{p(x_i, y_j)}{p(x_i)p(y_j)}
    \end{equation}
    互信息与相关性 ρ 相关, ρ 用来描述线性相关性,互信息用来描述非线性相关性,其中:
    \begin{equation}
    \rho = \frac{cov(x,y)}{\sqrt{x}\sqrt{y}}
    \end{equation}

    \subsection{条件熵}
    条件熵实际上是联合熵与熵的差集,也可表示为熵与互信息的差集,具体如下:
    \begin{equation}
    H(X|Y) = H(X,Y) - H(Y) = H(X) - I(X, Y)
    \end{equation}

    \subsection{相对熵}
    相对熵用来描述两个分布之间的差异。
    \begin{equation}
    KL(p||q) = \sum^n_{i=1}p(x_i)log\frac{p(x_i)}{q(x_i)}
    \end{equation}
    
    其中,p,q表示两个分布,其中:
    $KL(p||q)≠KL(q||p)$
    KL散度越大,两个分布间的差异越明显,并且:
    $KL(p||q) ≥ 0$

    \subsection{交叉熵} 
    主要用于度量两个概率分布间的差异性信息, 在分类任务中常用做目标函数
    为什么KL散度用来衡量两个分布的相似程度,交叉熵也用来衡量,请往后看,交叉熵的公式为:
    \begin{equation}
    H(p, q) = \sum^n_{i=1}p(x_i)log(\frac{1}{q(x_i)})
    \end{equation}

\end{document}