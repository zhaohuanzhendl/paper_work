\documentclass[UTF8]{ctexart}
   \usepackage{amsmath}
   \usepackage{amssymb}
\begin{document}
    \section{Rige Regression} 
        \subsection{线性回归基本计算公式} $y=w^Tx$.
        \subsection{带正则的脊回归基本计算公式} 
        \begin{equation}
            %t_\text{pix}=\frac{\Delta T}{N_\text{pix}}\times N_\text{det}.
        J(w) = (y-Xw)^T(y-Xw) + \lambda{\parallel{w}\parallel}^2
        \end{equation}
        \paragraph{求解推导过程}
        %\subparagraph{Hello Chairman Mao} is in the center of Tian'anmen Square\\
        其中X是一个样本矩阵$X$每一行是一个样本$y$是label向量
        \begin{equation}
        \begin{aligned}
          \nabla _w J(w) & = -X^Ty-X^Ty+2X^TXw+2\lambda w \\
          & =-2X^Ty+2(X^TX+\lambda I)w = 0
        \end{aligned}
        \end{equation}
        求解$w$后的结果可得:
        \begin{equation}
        w = (X^TX+\lambda I)^-1X^Ty
        \end{equation}
        
    \section{Kernel Rige Regression}
        \subsection{核脊回归}
        这个形式因为有一项X没有办法写成内积的形式,所以我们需要将其转换,这里用到一个Matrix inversion lemma (MLAPP Page 117)的引理:
        Matrix inversion lemma: 考虑一个一般的矩阵分割 
        \begin{equation}
        M = \left(
        \begin{array}{cc}
        E & F  \\
        G & H
        \end{array}
        \right)
        \end{equation}
        假设E和H都是可逆的于是有:
        \begin{equation}
        \begin{aligned}
        (E-FH^-1G)^{-1}&=E^{-1}+E^{-1}F(H-GE^{-1}F)^{-1}GE^{-1} \\
        (E-FH^-1G)^{-1}FH^{-1}&=E^{-1}F(H-GE^{-1}F)^{-1}
        \end{aligned}
        \end{equation}
        对上述第二条公式w的最优解进行化简, 令$H\triangleq \lambda^{-1}I, F\triangleq X^T, G\triangleq-X, E\triangleq I$ 得到如下等式:
        \begin{equation}
        \begin{aligned}
          (E-FH^-1G)^{-1}FH^{-1} & = (I+X^T\lambda^{-1}X)^{-1}X^T\lambda^{-1} \\
          & =(\lambda I+X^TX)^{-1}X^T
        \end{aligned}
        \end{equation}
        使用上述公式对$w$结果进行变换如下:
        \begin{equation}
        \begin{aligned}
          w & = (X^TX+\lambda I)^-1X^Ty \\
          & =X^T(\lambda I+XX^T)^{-1}y
        \end{aligned}
        \end{equation}
        接下来对上述公式进行kernel化,将w写成向量求合的形式 令
        \begin{equation}
        \alpha \triangleq (K+\lambda I_{N})^{-1}y
        \end{equation}
        于是w可以写成 
        \begin{equation}
        w = X^T\alpha = \sum_{i=1}^{N}{\alpha_i x_i}
        \end{equation}
        通过上述式子可以看出 $w$其实是所有样本的一个加权平均, 当新来一个待预测样本时 新标签值为
        \begin{equation}
        y^{*} = w^Tx^{*} = \sum_{i=1}^{N}{\alpha_i x_{i}^Tx^{*}} = \sum_{i=1}^{N}{\alpha_i k(x^{*}, x_i)}
        \end{equation}

\end{document}
